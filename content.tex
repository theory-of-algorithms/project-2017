%!TEX root = project.tex
\noindent
The following document contains the instructions for \projectname{} for \modulename{}.
The project will be worth 30\% of your mark for this module.
You must use GitHub to manage the development of your project.

You are required to develop a single-page web application(SPA)~\cite{singlepageapps} written in the programming language Python~\cite{pythonorg} using the Flask framework~\cite{pythonflask}.
You must devise an idea for a web application, write the software, write documentation explaining how the application works, and write a short user guide for it.

\subsection*{GitHub}
GitHub must be used to manage the development of the software.
You might also try using GitHub Issues~\cite{githubissues}.
GitHub issues allow collaborators to track project progress through bug reports, milestones and labels.

You will have the opportunity during the weekly timetabled computer labs to work on the project, and to ask the lecturer for advice.
The lecturer will add issues and comments to your repository to record these interactions.
It is likely that these interactions will heavily affect the marking of the project.

\section*{Technology}
You can use any combination of the following technologies in your project.
However, note that the main part of the application must be written in Python.
Any other technologies including libraries and frameworks you wish to use must be agreed to by the lecturer.

\begin{description}
\item[Languages ---] Python, HTML, JavaScript, CSS, Typescript, Less, Sass
\item[Libraries ---] Anulgar.js, jQuery, React, Bootstrap, Skeleton, Ember.js.
\item[Frameworks ---] Flask, Pyramid.
\item[Databases ---] Any SQL RDBMS, CouchDB, MongoDB, Neo4j, Redis.
\end{description}

\section*{Submission}
Your GitHub repository, along with the Issues tracker and all other GitHub facilities, will form the main submission of the project.

\subsection*{Quality}
A good project submission will demonstrate a well-managed, steady effort to create a useful and easy-to-use web application.
The code base will be well thought-out.

\subsection*{Team-work}
Students may work on this project as a team, subject to approval from the lecturer.
Note that team submissions must reflect a team effort, and may be marked to a higher standard.
Team members will still, however, receive an individual mark which will mainly be determined by the individual's contributions to the team repository.
Thus, it is important that team members ensure they make sufficient contributions to the repository in their own name.

\subsection*{Student Conduct}
You should familiarise yourself with GMIT's code of student conduct~\cite{gmitconduct} and the policy on plagiarism~\cite{gmitplagiarism}.
In particular, note two things.
First, students are expected to treat other students and staff politely and with courtesy.
Second, it is assumed that all work you submit is being presented as your own work, unless referenced otherwise.

\subsection*{Marking scheme}
The following marking scheme will be used to mark the project.
Students should note, however, that in certain circumstances the examiner's overall impression of the project may influence marks in each individual component.

\begin{center}
\begin{tabular}{x{1cm}p{2.5cm}p{9cm}}
\toprule
40\% & Programming & Well-written code; logical directory structure; extensive commenting; extensive git commits.\\
\midrule
40\% & Architecture & Good separation of concerns; clear API. \\
\midrule
20\% & User Experience & Nice flow through the application; clear user interface. \\
\bottomrule
\end{tabular}
\end{center}

\subsection*{Expected standard}
Please note that this is a level 7 module.
You should be aware that the standard required for submissions at level 8 (fourth year) is higher than at level 7 (third year), which in turn is higher than at level 6 (first and second year).
Significant effort is made to ensure that the standard is fair and consistent across third level institutes, both nationally and internationally.
The standard we set for modules in computing is informed by Quality and Qualifications Ireland's Award Standard for Computing~\cite{qqicomputing}.
Below is a particularly relevant selection of the learning outcomes contained in that document.

\subsubsection*{\small Level 8 (Year 4)}
\emph{
The learner will be able to:
\begin{itemize}
\item describe the limitations of some current computing theories.
\item evaluate information through online research.
\item model and design complex computer-based systems in a way that demonstrates comprehension of the trade-off involved in design choices.
\item demonstrate mastery of a complex and specialised area of skills and tools;
\item manage one's own learning and development, including time management and organisational skills.
\item manage a computer-based project throughout all stages of the life-cycle.
\item apply quality concepts to products and processes of own work.
\end{itemize}
}

\subsubsection*{\small Level 7 (Year 3)}
\emph{
The learner will be able to:
\begin{itemize}
\item integrate concepts learned across a variety of subject areas.
\item identify relevant material on a given topic from available information sources.
\item succinctly present rational and reasoned arguments to a range of audiences.
\item develop innovative solutions to pragmatic situations.
\item recognise the suitability of a given solution to a problem.
\item apply knowledge learned in new situations.
\end{itemize}
}
\subsubsection*{\small Level 6 (Years 1 and 2)}
\emph{
The learner will be able to:
\begin{itemize}
\item describe, recognise and apply best practices in computing.
\item apply knowledge in a practical setting under supervision.
\item demonstrate the capacity to learn new knowledge and skills.
\item use troubleshooting strategies and techniques in correcting a variety of computer hardware and software problems.
\item implement computer based systems solutions to well-defined problems.
\end{itemize}
}
